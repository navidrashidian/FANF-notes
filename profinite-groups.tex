\documentclass[12pt]{article}

\usepackage{enumitem}
\usepackage[hidelinks]{hyperref}

\usepackage[backend=bibtex, style=numeric, citestyle=numeric-comp]{biblatex}

\usepackage{amsmath,amssymb,amsfonts,amsthm}

\usepackage{tikz-cd}

\newtheorem{thm}{Theorem}[section]
\newtheorem{prop}[thm]{Proposition}
\newtheorem{lem}[thm]{Lemma}

\theoremstyle{definition}
\newtheorem{dfn}[thm]{Definition}
\newtheorem{exm}[thm]{Example}

\theoremstyle{remark}
\newtheorem{rem}[thm]{Remark}

\newcommand{\catvar}[1]{\mathcal{#1}}
\newcommand{\CC}{\catvar{C}}
\newcommand{\DD}{\catvar{D}}
\newcommand{\PP}{\catvar{P}}
\newcommand{\II}{\catvar{I}}
\newcommand{\JJ}{\catvar{J}}

\newcommand{\catname}[1]{\textup{\sffamily\textbf{#1}}}
\newcommand*{\Set}{\catname{Set}}
\newcommand*{\Grp}{\catname{Grp}}
\newcommand*{\Top}{\catname{Top}}
\newcommand*{\TopGrp}{\catname{TopGrp}}

\newcommand{\zee}{\mathbb{Z}}
\newcommand{\zeehat}{\hat{\mathbb{Z}}}

\newcommand*{\nor}{\mathcal{N}}

\newcommand{\Obj}[1]{\textup{Obj}(#1)}
\newcommand{\Mor}[1]{\textup{Mor}(#1)}
\newcommand{\Hom}[2]{\textup{Hom}(#1,#2)}

\newcommand{\gcirc}{G^\circ}

\addbibresource{profinite-groups.bib}

\title{Profinite Groups}
\author{Navid Rashidian}
\date{}

\begin{document}

    \maketitle

    \section{Category Theory}

    \begin{dfn}
        A \emph{category} $\mathcal{C}$ consists of collections $\Obj{\CC}$ and $\Mor{\CC}$, respectively called the objects and morphisms of $\CC$ such that
        \begin{itemize}
            \item there are functions $s\colon\Mor{\CC}\to\Obj{\CC}$ and $t\colon\Mor{\CC}\to\Obj{\CC}$  assigning to each morphism a source and a target;
            \item for each object $X\in\Obj{\CC}$ there is a distinguished morphism $\textup{id}_X\in\Mor{\CC}$ such that $s(\textup{id}_X)=t(\textup{id}_X)=X$; and
            \item for each pair of morphisms $f,g$ such that $t(f)=s(f)$ there is a morphism $g\circ f$ with $s(g\circ f)=s(f)$ and $t(g\circ f)=t(g)$ called their composition;
        \end{itemize}
        satisfying the further conditions that $(f\circ g)\circ h=f\circ(g\circ h)$ and $\textup{id}_X\circ f=f$ and $f\circ\textup{id}_X=f$ wherever these expressions make sense.
    \end{dfn}

    \begin{rem}
        We write $f\colon X\to Y$ to denote a morphism $f\in\Mor{\CC}$ with $s(f)=X$ and $t(f)=Y$.
    \end{rem}

    \begin{dfn}
        Let $\CC$ be a category. If both $\Obj{\CC}$ and $\Mor{\CC}$ are sets, The category $\CC$ is called a \emph{small category}. $\CC$ is called \emph{locally small} if for every objects $X,Y\in\Obj{\CC}$ the collection of morphisms $f\in\Mor{\CC}$ with $s(f)=X$ and $t(f)=Y$ is a set. In a locally small category for objects $X,Y\in\Obj{\CC}$ we define the \emph{hom-set of $X$ and $Y$} as
        $$
            \textup{Hom}_\CC(X,Y)=\{f\in\Mor{\CC}:s(f)=X\ \textup{and}\ t(f)=Y\}
        $$
    \end{dfn}

    \begin{dfn}
        Let $\CC$ be a category and $f\colon X\to Y$ a morphism in $\CC$. We call $f$ an \emph{isomorphism} if there is a morphism $g\colon Y\to X$ such that $g\circ f=\textup{id}_X$ and $f\circ g=\textup{id}_Y$. If such morphisms exist, we call $X$ and $Y$ \emph{isomorphic}.
    \end{dfn}

    \begin{dfn}
        Let $\CC$ and $\DD$ be categories. A \emph{(covariant) functor} $F\colon\CC\to\DD$ consists of maps $\Obj{\CC}\to\Obj{\DD}$ and $\Mor{\CC}\to\Mor{\DD}$ such that
        \begin{itemize}
            \item for every $f\in\Mor{\CC}$ we have $s(F(f))=F(s(f))$ and $t(F(f))=F(t(f))$;
            \item for every $X\in\Obj{\CC}$ we have $F(\textup{id}_X)=\textup{id}_F(X)$; and
            \item for every pair of morphisms $f,g\in\Mor{\CC}$ with a defined composition $F(g\circ f)=F(g)\circ F(f)$.
        \end{itemize}
    \end{dfn}

    \begin{dfn}
        Suppose $\CC$ is a category. The \emph{opposite category} denoted by $\CC^{\textup{op}}$ is the category with the same objects and morphisms as $\CC$ such that $s_{\CC^{\textup{op}}}(f)=t_{\CC}(f)$ and $t_{\CC^{\textup{op}}}(f)=s_{\CC}(f)$ and $g\circ_{\CC^\textup{op}} f=f\circ_\CC g$.
    \end{dfn}

    \begin{dfn}
        A \emph{contravariant function} $F\colon\CC\to\DD$ is a covariant functor $F\colon\CC^\textup{op}\to\DD$.
    \end{dfn}

    \begin{exm}
        A \emph{preorder} is a small categoy $\PP$ such that for every objects $X,Y\in\Obj{\PP}$ we have $\#\textup{Hom}_\PP(X,Y)\leq1$. If for every $X,Y\in\Obj{\PP}$ there is an object $Z\in\Obj{\PP}$ with morphisms $f\colon X\to Z$ and $g\colon Y\to Z$, the preorder $\PP$ is called a \emph{directed set}. For $I,J\in\Obj{\PP}$ we write $I\leq J$ if $\#\textup{Hom}_\PP(X,Y)=1$.
    \end{exm}

    \begin{exm}
        A \emph{partially ordered set (poset)} is a small categoy $\PP$ such that for every pair of objects $X,Y\in\Obj{\PP}$ we have $$\#\big(\textup{Hom}_\PP(X,Y)\cup\textup{Hom}_\PP(Y,X)\big)\leq1$$. For posets $\PP_1$ and $\PP_2$ an order-preserving (-reversing) function from $\PP_1$ to $\PP_2$ is exactly a covariant (contravariant) functor $F\colon\PP_1\to\PP_2$.
    \end{exm}

    \begin{exm}
        Sets and functions constitute a category denoted by $\Set$. Groups and group homomorphisms constitute a category denoted by $\Grp$. Topological spaces and continuous functions constitute a category denoted by $\Top$. Topological groups and continuous homomorphisms constitute a category denoted by $\TopGrp$. All of these categories are locally small but none is small.
    \end{exm}

    \begin{exm}
        The category $\catname{2}$ is the category consisting of two objects and only the required identity morphisms:
            \[\begin{tikzcd}
                \bullet && \bullet
            \end{tikzcd}\]
        A funtor $F\colon\catname{2}\to\CC$ is just a pair of objects from $\Obj{\CC}$. Similarly, for every cardinality $\kappa$ we can define a discrete category with cardinality $\kappa$ to serve as an index set.
    \end{exm}

    \begin{exm}\label{exm:pod}
        The following diagram describes a category consisting of three objects and two morphisms (in addition to the required identity morphisms):
        \[\begin{tikzcd}
            && \bullet \\
            \\
            \bullet && \bullet
            \arrow[from=1-3, to=3-3]
            \arrow[from=3-1, to=3-3]
        \end{tikzcd}\]
        Call this category $\catvar{J}$. A functor $F\colon \JJ\to\CC$ describes a diagram of the following shape in the category $\CC$:
        \[\begin{tikzcd}
            && X \\
            \\
            Y && Z
            \arrow["f", from=1-3, to=3-3]
            \arrow["g", from=3-1, to=3-3]
        \end{tikzcd}\]
        This justifies us calling a functor from a small category a diagram.
    \end{exm}

    \begin{dfn}
        Let $F\colon\JJ\to\CC$ be a diagram. A \emph{cone} over $F$ is a an object $X\in\Obj{\CC}$ and a collection of morphisms $\alpha_J\colon X\to F(J)$ indexed by objects $J\in\Obj{\JJ}$ such that for every morphism $f\colon J\to J'$ we have $F(f)\circ \alpha_{J}=\alpha_{J'}$.
    \end{dfn}

    \begin{exm}
        A diagram $F\colon\catname{2}\to\CC$ is just a pair of discrete objects $X,Y\in\Obj{\CC}$. A cone over $F$ is just an object $A\in\Obj{\CC}$ and a pair of morphisms $f\colon A\to X$ and $g\colon A\to Y$.
    \end{exm}
    
    \begin{exm}
        Recall the category $\JJ$ from example \ref{exm:pod}. A cone over $F:\JJ\to\CC$ is fully described by an object $A\in\Obj{\CC}$ and morphisms $\alpha\colon A\to X$ and $\beta\colon A\to Y$ such that the following diagram commutes:
        \[\begin{tikzcd}
            A && X \\
            \\
            Y && Z
            \arrow["\alpha", from=1-1, to=1-3]
            \arrow["\beta", from=1-1, to=3-1]
            \arrow["f", from=1-3, to=3-3]
            \arrow["g", from=3-1, to=3-3]
        \end{tikzcd}\]
        (Note that we don't need to explicitly describe the morphism $A\to Z$ included in the cone.)
    \end{exm}

    \begin{dfn}
        Let $F\colon\JJ\to\CC$ be a diagram. A \emph{limit of $F$} is a cone $\langle L,\alpha_J\rangle$ over $F$ such that for every cone $\langle N,\beta_J\rangle$ over $F$ there is a unique morphism $u\colon N\to L$ such that for every $J\in\Obj{\JJ}$ we have $\beta_J=\alpha_J\circ u$.
    \end{dfn}

    \begin{rem}
        Let $\langle L,\alpha_J\rangle$ and $\langle L',\beta_J\rangle$ be limits of $F:\JJ\to\CC$. By definition there are morphisms $u\colon L'\to L$ and $u'\colon L\to L'$ such that for every $J\in\Obj{\JJ}$ we have $\beta_J=\alpha_J\circ u$ and $\alpha_J=\beta_J\circ u'$. It follows that for every $J\in\Obj{\JJ}$ we have $\alpha_J=\alpha_J\circ(u\circ u')$. According to the definition of limit only one function should play such role, so $u\circ u'=\textup{id}_L$ and similarly $u'\circ u=\textup{id}_{L'}$. Therefore $L$ and $L'$ are isomorphic. As a folk who have internalized Leibniz's principle of the Identity of Indiscernibles, this makes us usually talk about \emph{the} limit of a diagram.
    \end{rem}

    \begin{exm}
        Consider a diagram $F\colon\catname{2}\to\Set$ consisting of sets $X$ and $Y$. It is easy to verify that the cartesian product $X\times Y$ with the obvious projection maps $\pi_X$ and $\pi_Y$ is a limit of $F$. The same is true if we replace $\catname{2}$ with another discrete category with the requisite cardinality serving as an index set.
    \end{exm}

    \begin{exm}
        Let $\II$ be a discrete category serving us an index set. A functor $F\colon\II\to\Top$ is a family $\big\langle\langle X_I,\tau_I\rangle\big\rangle$ of toplogical spaces indexed by $\II$. Let $X=\langle\prod X_I,\tau\rangle$ and the obvious projection mappings $\pi_{X_I}$ the cartesian product of the $X_I$'s with the usual product topology having the collection of all $(\pi_{X_I})^{-1}(U)$'s for some $U\in\tau_I$ as a subbasis. Since continuous maps are set maps it is obvious that for any given cone $\langle Y,\alpha_{X_I}\rangle$ there is only one possible candidate defined by $u\colon y\mapsto (\alpha_{X_I}(y))$ as the required morphism $Y\to X$. Hence, we only need to show that $u$ is continuous if the $\alpha_{X_I}$'s are. This last fact is obvious since every elements of the subbasis according to which $\tau$ is defined is in the form of $(\pi_{X_I})^{-1}(U)$ and $u^{-1}((\pi_{X_I})^{-1}(U))=(\alpha_I)^{-1}(U)$ which is gauranteed to be open if $\alpha_I$ is continuous.

        If we instead equip $\prod X_I$ with the box topology generated by sets in the form of $\prod U_I$ such that every $U_I$ is a member of $\tau_I$, the map $y\mapsto (\alpha_{X_I}(y))$ will not be necessarily continuous. Moreover the cartesian product of compact topological spaces equipped with the box topology is not necessarily compact, while their cartesian product equipped with the product topology is necessarily so. (See Wikipedia: Box Topology.)
    \end{exm}

    \section{Projective Systems and Projective Limits}

    \begin{dfn}
        Let $\II$ be a preordered set. A contravariant functor $F\colon\II\to\CC$ is called a \emph{projective} (or \emph{inverse}) \emph{system}.
    \end{dfn}

    \begin{rem}
         Since in a preorder there is at most one morphism between any two objects, in a given projective system we could denote the possible morphism $X_J\to X_I$ as $\varphi_{IJ}$ without ambiquity (note the difference in order). We will usually denote a projective system by $\langle X_I,\varphi_{IJ}\rangle$. We will also usually describe the family of maps in a cone on $F$ as a family of maps compatible with the projective system.
    \end{rem}

    \begin{dfn}
        Let $F\colon \II\to\CC$ be a projective system. The limit of $F$ is called the \emph{projective limit} or \emph{inverse limit} of this system and is denoted by $\varprojlim X_I$.
    \end{dfn}

    \begin{exm}
        In the category $\Set$ for a given projective system $\langle X_I,\varphi_{IJ}\rangle$ the following explicit construction equipped with the obvious projection maps $\pi_I$ describes a projective limit:
        $$
            \varprojlim X_I = \{(x_I)\in\prod X_I : \varphi_{IJ}(x_J)=x_I\ \textup{for every}\ I,J\in\Obj{\II}\}
        $$
        For a family of maps $\varphi_I\colon Y\to X_I$ compatible with the projective system, the map $x\mapsto (\varphi_I(x))$ is clearly the unique map $H\to\varprojlim X_I$ described in the definition of limit. The fac that $(\varphi_I(x))$ falls inside $\varprojlim X_I$ is gauranteed by the compatibility of the set of maps with the projective system.
    \end{exm}

    \begin{exm}\label{exm:emp}
        Consider the following projective system in $\Set$:
        \[\begin{tikzcd}
            && {\{0\}} \\
            \\
            {\{1\}} && {\{0,1\}}
            \arrow[hook, from=1-3, to=3-3]
            \arrow[hook, from=3-1, to=3-3]
        \end{tikzcd}\]
        This projective system is modelled on a preordered set that is not directed. It is easy to verify that its limit is empty.
    \end{exm}

    \begin{exm}
        Let $\langle G_I,\varphi_{IJ}\rangle$ be a projective system in the category $\Grp$. Let
        $$
             G = \{(x_I)\in\prod G_I : \varphi_{IJ}(x_J)=x_I\ \textup{for every}\ I,J\in\Obj{\II}\}
        $$
        and $\pi_I$'s the obvious projection maps. The identity element of $\prod G_I$ is clearly in $G$ and hence $G$ is necessarily non-empty. Also it is easy to verify that $G$ is a group. Finally, for every group $H$ and set of morphisms $\psi_I\colon H\to G_I$ we have, the map $u\colon H\to G$ defined by $x\mapsto (\psi_I(x))$ is the sole group homomorphism satisfying the desired property in the definition of projective limits. Hence, $\varprojlim G_I=G$. 
    \end{exm}

    We saw in Example \ref{exm:emp} that the projective limit of a projective system need not be non-empty. The following theorem gives us sufficient conditions for having a non-empty projective limit of finite sets:
    
    \begin{prop}
        Let $\II$ be a directed set and $\langle X_i,\varphi_{IJ}\rangle$ a projective system modelled on $\II$. Let $X=\varprojlim X_I$. Then
        \begin{enumerate}[label=(\roman*)]
            \item If all $X_I$'s are non-empty, then $X$ is non-empty.
            \item For each $I\in\Obj{\II}$ we have
            $$ \pi_I(X) =\bigcap_{I\leq J}\varphi_{IJ}(X_J) $$
        \end{enumerate}
    \end{prop}

    \begin{proof}
        See \cite[Proposition 1-11]{FANF1999}.
    \end{proof}

    \section{Profinite Groups}

    \begin{dfn}
        A topological group $G$ is called a \emph{profinite group} if there is a projective system $\langle G_I,\varphi_{IJ}\rangle$ such that $G$ is the projective limit of this projective system in the category $\catname{TopGrp}$. The topology on $G$ is called the \emph{profinite topology}.
    \end{dfn}

    The following sections introduce some examples of pro-finite groups.

    \subsection{\texorpdfstring{$p$}{p}-adic Numbers}

    % \begin{dfn}
    %     Let $G$ be a topological group. A \emph{Cauchy sequence} in $G$ is a sequence $(x_i)$ of elements of $G$ such that, for any neighborhood $U$ of $0$, there is a positive integer $N$ such that for every $m,n\geq N$ we have $x_m-x_n\in U$. Two Cauchy sequences $(x_i)$ and $(y_i)$ are \emph{equivalent} if $x_i-y_i\to 0$ in $G$.
    % \end{dfn}
    
    % \begin{proof}
    %     \begin{enumerate}[label=(\alph*)]
    %         \item Reflexivity is trivial. Symmetry follows from the continuity of $-\colon G\to G$. For transitivity, note that every neighborhood $U$ of $0$ contains a neighborhood $V$ of $0$ such that $V+V\subset U$ (\cite[Proposition 1-1(i)]{FANF1999}) and then let $N$ be an integer such that for $i>N$ we have $x_i-y_i\in V$ and $y_i-z_i\in V$. It follows that for $i>N$ we have $x_i-z_i\in U$.
    %         \item Let $(x_i)$ and $(y_i)$ be Cauchy sequences. Let $U$ be a neighborhood of $0$. There is a neighborhood $V$ of $0$ such that $V+V\subset U$. There is a positive integer $N$ such that for every $n>N$ we have $x_i,y_i\in V$ and $x_i+y_i\in U$.
    %         \item Note that $(x_i+x'_i)-(y_i+y'_i)=(x_i-y_i)+(x'_i-y'_i)$ and use the fact that for every neighborhood $U$ of $0$ there is a neighborhood $V$ of $0$ such that $V+V\subset U$ to show that $(x_i+x'_i)-(y_i+y'_i)\to 0$.
    %     \end{enumerate}
    % \end{proof}
    
    \begin{dfn}
        Let $R$ be an integral domain. An \emph{absolute value on $R$} is a function $|.|\colon R\to \mathbb{R}$ such that for every $x,y\in R$ we have
        \begin{enumerate}[label=(\alph*)]
            \item $|x|\geq 0$;
            \item $|x|=0$ iff $x=0$;
            \item $|xy|=|x||y|$; and
            \item $|x+y|\leq |x|+|y|$.
        \end{enumerate}
    \end{dfn}

    \begin{dfn}
        A \emph{metric space} is a set $M$ equipped with a function $d\colon M\times M\to \mathbb{R}$ such that for every $x,y,z\in M$ we have
        \begin{enumerate}[label=(\alph*)]
            \item $d(x,x)=0$;
            \item if $x\neq y$ then $d(x,y)>0$;
            \item $d(x,y) = d(y,x)$; and
            \item $d(x,z)\leq d(x,y)+d(y,z)$.
        \end{enumerate}
    \end{dfn}

    \begin{prop}
        Let $R$ be an integral domain and $|.|$ an absolute value defined on $R$. Then $d(x,y)=|x-y|$ is a metric on $R$.
    \end{prop}

    \begin{dfn}
        For any point $x\in M$ and any real number $r>0$ \emph{the open ball of radius $r$ around $x$} is defined by
        $$B_r(x)=\{y\in M : d(x,y) < r\}$$
        The open balls form the basis for the \emph{metric topology on $M$}.
    \end{dfn}

    \begin{dfn}
        A sequence $(x_n)$ of points in a metric space $M$ is a \emph{Cauchy sequence} if for every $r>0$ there is a positive integer $N$ such that for every pair of positive integers $m,n\geq N$ we have $d(m,n)<r$. Two Cauchy sequences $(x_n)$ and $(y_n)$ are called \emph{equivalent} if $x_n-y_n$ converges to zero.
    \end{dfn}

    \begin{prop}
        The equivalence defined on Cauchy sequences is an equivalence relation.
    \end{prop}
    
    \begin{prop}
        Assume that $+$ and $\cdot$ are continuous with respect to the metric topology.
        \begin{enumerate}[label=(\alph*)]
            \item If $(x_i)$ and $(y_i)$ are Cauchy, then so is $(x_i+y_i)$ and $(x_iy_i)$.
            \item Suppose $(x_i)\sim(y_i)$ and $(x'_i)\sim(y'_i)$. Then $(x_i+x'_i)\sim(y_i+y'_i)$ and $(x_ix'_i)\sim(y_iy'_i)$.
        \end{enumerate}
    \end{prop}

    \begin{dfn}
        A \emph{complete metric space} is a metric space in which every Cauchy sequence converges to a point in the space.
    \end{dfn}

    \begin{dfn}
        Let $M$ be a metric space and $\tilde{M}$ the set of Cauchy sequences with members in $M$. Then $\hat{M}=\tilde{M}/\sim$ is called \emph{the completion of $M$}. The distance function on $\tilde{M}$ is defined by $\lim_{n}d(x_n,y_n)$.
    \end{dfn}

    \begin{dfn}
        Let $p$ be a fixed prime. The \emph{$p$-adic valuation} is the function $v_p\colon\mathbb{Z}\to\mathbb{Z^+}\cup\{\infty\}$ which assigns to each positive integer $n$ the unique $e$ such that $n=p^em$ and $p\nmid m$ and assigns $\infty$ to 0. The \emph{$p$-adic absolute value} is the function $|.|_p\colon\mathbb{Z}\to \mathbb{R}$ defined by $|n|_p=p^{-v_p(n)}$.
    \end{dfn}

    \begin{prop}
        The $p$-adic absolute value is an absolute value on $\mathbb{Z}$.
    \end{prop}

    \begin{exm}
        Integers are not complete under the $p$-adic metric. The sequence $(p^n+1)$ is clearly Cauchy but doesn't converge to any integer. 
    \end{exm}

    \begin{dfn}
        The group $\mathbb{Z}_p$ called the group of \emph{$p$-adic integers} is the completion of $\mathbb{Z}$ with respect to the $p$-adic metric.
    \end{dfn}

    We now want to prove that $\mathbb{Z}_p$ defined as the completion of $\mathbb{Z}$ with respect to the $p$-adic metric is the inverse limit of the projective system $\mathbb{Z}/p^e\mathbb{Z}$ with the natural projections $\varphi_{n(n+1)}\colon \mathbb{Z}/p^{n+1}\mathbb{Z}\to \mathbb{Z}/p^{n}\mathbb{Z}$.

    \begin{lem}
        The following are equivalent:
        \begin{enumerate}[label=(\alph*)]
            \item $|x_n-x_m|_p\leq p^{-e}$.
            \item $v_p(x_n-x_m)\geq e$.
            \item $x_n\equiv x_m \mod p^e$.
        \end{enumerate}
    \end{lem}

    \begin{proof}
        Expand the definitions.
    \end{proof}

    \begin{lem}
        Consider $\mathbb{Z}$ with the metric induced by the $p$-adic absolute value for a fixed prime $p$. Let $(x_n)$ be a Cauchy sequence. For every positive integer $e$ the remainder of $x_i$'s modulo $p^e$ eventually stops. Moroever, if $(x_n)$ and $(y_n)$ are equivalent Cauchy sequences, their remainders modulo $p^e$ stop on the same number. Therefore, remainder modulo $p^e$ defines a function $\varphi_e\colon \mathbb{Z}_p\to \mathbb{Z}/p^e\mathbb{Z}$. The function $\varphi_e$ is a continuous ring homomorphism.
    \end{lem}

    \begin{proof}
        Let $r=p^{-e}$. There is a positive integer $N$ such that for every pair of positive integers $m,n> N$ we have $x_m-x_n<r$ or equivalently
        $$x_m\equiv x_n \mod p^e$$
        A similar argument proves the other claim. That $\varphi_e$ respects addition and multiplication is obvious. For continuity it suffices to prove that the pre-image of singletons is open. Now, consider $\varphi_e((x_n))=\varphi_e((y_n))$. Then
        $$d((x_n),(y_n))=\lim_nd(x_n,y_n)\leq p^{-e}$$
        Moreover, it is clear that the converse implication is also true. Hence, the pre-image of $\varphi_e((x_n))$ is exactly the open ball of radius $p^{-e}$ around $(x_n)$.
    \end{proof}

    \subsection{Profinite Integers}

    \begin{dfn}
        Consider the family of rings $\zee/n\zee$ for every positive integer $n$. For every $m,n$ such that $m\mid n$ there is a canonical projection $\varphi_{mn}\colon\zee/n\zee\to\zee/m\zee$. It is clear that this family of sets and projections form a projective system in the category of topological rings. The projective limit of this projective system is called \emph{profinite completion of integers} and is denoted by $\zeehat$.
    \end{dfn}

    \begin{prop}[Chinese Remainder Theorem]
        Let $n_1,\cdots,n_k$ be coprime integers and $N=n_1\cdots n_k$. The map
        $$x \mod N \mapsto (x \mod n_1,\cdots,x \mod n_k)$$
        is an isomorphism of rings
        $$\mathbb{Z}/N\mathbb{Z}\cong \mathbb{Z}/n_1\mathbb{Z}\times\cdots\times\mathbb{Z}/n_k\mathbb{Z}$$
    \end{prop}

    \begin{prop}
        $\zeehat\cong\prod_{p\textup{ prime}}\zee_p$.
    \end{prop}

    \begin{proof}
        Let $(x_n)\in\zeehat$ be a coherent sequence of elements of all $\zee/n\zee$s. Let $n=p_1^{v_{p_1}(n)}\cdots p_k^{v_{p_k}(n)}$ be a positive integer. According to the Chinese Remainder Theorem, for any integer $x$ the remainder of $x$ modulo all $p_i^{{v_i}(n)}$'s completely determines its remainder moduolo $n$. Hence $(x_n)$ is fully determined by the $x_i$'s such that $i=p^e$ for a prime $p$ and a positive integer $e$. This induces a bijection between $\zeehat$ and $\prod_{p\textup{ prime}}\zee_p$. That this bijection is a ring homomorphism follows from the fact that in both cases the arithmetic operations are inherited from the arithmetic operations on $\zee$.
    \end{proof}

    \begin{exm}
        The projections $\varphi_{mn}$ map units of $\zee/n\zee$ to units in $\zee/m\zee$. Therefore the groups $(\zee/n\zee)^\times$ for all positive integers $n$ and canonical projections $\varphi_{mn}$ form a projective system. Let $(\zeehat)^\times$ denote the inverse limit of this system. It is clear that $(\zeehat)^\times$ is the group of units of $\zeehat$. Also from the fact that $\zeehat\cong\prod_{p\textup{ prime}}\zee_p$ it can be easily deduced that $(\zeehat)^\times\cong\prod_{p \textup{ prime}}(\zee_p)^\times$.
    \end{exm}

    \section{The topology of profinite groups}
    
    The following propositions states some of the most important global properties of profinite groups.

    \begin{prop}\label{prop:top}
        Let $G$ be a profinite group, given as a limit of the projective system $\langle G_I,\varphi_{IJ}\rangle$. Then,
        \begin{enumerate}[label=(\alph*)]
            \item $G$ is Hausdorff;
            \item $G$ is a closed subset of the direct product $\prod G_I$; and
            \item\label{prop:top:compact} $G$ is compact.
        \end{enumerate}
    \end{prop}
    \begin{proof}
        \begin{enumerate}[label=(\alph*)]
            \item The direct product of Hausdorff spaces is Hausdorff. Hence, $G$ is Hausdorff.
            \item The complement of $G$ in $\prod G_I$ is the set of sequences in $\prod G_I$ that are not coherent, \emph{i.e.} sequences $(x_I)$ such that for some $I\leq J$ we have $\varphi_{IJ}(x_J)\neq(x_I)$. Now, fix some $I\leq J$ and some $x_J\in G_J$. The set of sequences $(x_I)\in\prod G_I$ whose element with index $J$ equals $x_J$ and furthermore satisfy the condition $\varphi_{IJ}(x_J)\neq(x_I)$ is the intersection of the pre-images of $\{x_J\}\subset G_J$ and $G_I - \{\varphi_{IJ}(x_J)\}\subset G_I$ which is clearly open. Therefore,
            $$
            G^c=\bigcup_I\bigcup_{J\geq I}\{(g_I)\in\prod G_I:\varphi_{IJ}(x_J)\neq(x_I)\}
            $$
            is a union of open sets and hence open.
            \item The direct product $\prod G_I$ is compact. A closed subset of a compact space equipped with the subspace topology is compact.
        \end{enumerate}
    \end{proof}

    Another important topological property of profinite groups is being totally disconnected. Finally, we will see that profinite groups are exactly compact and totally disconnected topological groups. First, we need to review some definitions from general topology.

    \begin{dfn}
        A topological space $X$ is called \emph{connected} if whenever $X=U\cup V$ for non-empty open sets $U$ and $V$ then $U\cap V\neq\varnothing$. A subset $Y\subset X$ is called \emph{a connected subset} if it is a connected space with respect to the subspace topology.
    \end{dfn}

    \begin{prop}
        A topological space $X$ is connected iff for any two non-empty closed sets $U$ and $V$ such that $X=U\cup V$ we have $U\cap V\neq\varnothing$.
    \end{prop}
    \begin{proof}
        Let $X=U\cap V$ be a connected space and $U$ and $V$ closed sets. Suppose $U\cap V=\varnothing$. Then $U$ and $V$ are open, which results in a contradiction. The argument for the other direction is similar.
    \end{proof}

    We will take for granted that every point in a topological space is contained in a maximal connected subspaces:
    \begin{dfn}
        Let $X$ be a topological space and $x\in X$. The maximal connected subspace of $X$ that includes $x$ is called the \emph{connected component of $x$}. In a topological group the connected component of identity is denoted by $G^\circ$.
    \end{dfn}

    \begin{dfn}
        A topological space is called \emph{totally disconnected} if the connected component of each point is its singleton.
    \end{dfn}

    We have seen that a topological group is homogenous in the sense that for every $x,y\in X$ there is a homeomorphism (vid. $z\mapsto zx^{-1}y$) that maps $x$ to $y$. Since homeomorphisms bijectively map connected components to connected components it is easy to see that a topological space is totally disconnected if and only if $G^\circ=\{e\}$.

    \begin{lem}\label{lem:gcirc}
        In a topological group $G$ the subset $\gcirc$ is a normal subgroup which is contained in every open subgroup.
    \end{lem}

    \begin{proof}
        Let $x\in\gcirc$. Then $x^{-1}\gcirc$ is connected by homogeneity and includes identity. Therefore $x^{-1}\gcirc\subset\gcirc$ which implies $x^{-1}\in\gcirc$. This shows that $\gcirc$ is closed under inverses. Since $x^{-1}\in\gcirc$, we know $x\gcirc$ includes identity too and is connected by homogeneity. Hence, $x\gcirc\subset\gcirc$ which shows $\gcirc$ is closed under multiplication. Now, $x\gcirc x^{-1}$ is also connected and includes identity and therefore is contained in $\gcirc$. This shows that $\gcirc$ is a normal subgroup of $G$.

        Let $U$ be an open subgroup of $G$. Then $U\cap G^\circ$ is an open subgroup of $G^\circ$ according to the subspace topology. From elementary group theory we know that $\gcirc$ is partitioned into the cosets of $U\cap\gcirc$. Now, every coset of the form $x\cdot (U\cap G^\circ)$ must be open by translation invariance. Therefore we could partition $\gcirc$ into $U\cap\gcirc$ and the union of all other cosets
        $$
        V=\bigcup_{x\in\gcirc-U}x\cdot(U\cap\gcirc)
        $$
        which is also open according to the subspace topology. Since $U\cap\gcirc$ and $V$ are disjoint and $\gcirc$ is connected, one of them must be empty. Identity is in $U\cap\gcirc$, so $V$ is the empty one. It follows that $\gcirc=U\cap\gcirc$ or equivalently $\gcirc\subset U$.
    \end{proof}

    \begin{prop}\label{prop:tod}
        A topological group $G$ is totally disconnected.
    \end{prop}

    \begin{proof}
        It suffices to prove that $G^\circ=\{e\}$. Let $G=\varprojlim G_I$. For every $I$, the set $\pi^{-1}_I(\{e_I\})$ is an open subgroup of $G$. From Lemma \ref{lem:gcirc} it follows that $\gcirc\subset\pi^{-1}_I(\{e_I\})$. Identifying $G$ with the set of coherent sequences of members of $G_I$'s, it follows that if $(x_I)\in \gcirc$, then for every $I$ we have $x_I=e_I$, which completes our proof.
    \end{proof}

    \begin{lem}
        Let $\nor$ be the family of open normal subgroups of a compact topological group $G$. Then,
        \begin{enumerate}[label=(\alph*)]
            \item For each $N\in\nor$, the quotient group $G/N$ is compact and discrete, and hence finite.
            \item The relation $M\leq N$ if $N\subset M$ is a preorder on $\nor$. The maps $\varphi_{MN}\colon xN\mapsto xM$ create a projective system modelled on this directed set. Moreover, $G$ and the natural projections $\pi_N\colon G\to G/N$ form a cone on this system.
        \end{enumerate}
    \end{lem}

    \begin{proof}
        \begin{enumerate}[label=(\alph*)]
            \item Quotients of a compact space are compact. The quotient topology is the finest topology that makes the natural projection map continuous. For every $xN\in G/N$, we know that $xN$ is open in $G$ since $N$ is open and every translation of an open subset is open. Hence the discrete topology makes the natural projection map continuous. Compact spaces are discrete if and only if they are finite, therefore $G/N$ is finite.
            \item It is trivial that the relation is a preorder. To see that the maps $\varphi_{MN}$ create a projective system it suffices to note that for $L\leq M\leq N$ then $\varphi_{LM}\circ\varphi_{MN}=\varphi_{LM}$.
            
            To see that $G$ and the projection maps $\pi_N\colon G\to G/N$ form a cone on this projective system note that the following diagram commutes:

            \[\begin{tikzcd}
                & G \\
                \\
                {G/N} && {G/M}
                \arrow["{\pi_N}"', from=1-2, to=3-1]
                \arrow["{\pi_M}", from=1-2, to=3-3]
                \arrow["{\varphi_{MN}}"', from=3-1, to=3-3]
            \end{tikzcd}\]
        \end{enumerate}
    \end{proof}

    \begin{lem}\label{lem:sur}
        Let $G$ be a compact topological group and $$G'=\varprojlim_{N\in\nor} G/N$$
        There is a surjective continuous group homomorphism $\alpha\colon G\to G'$.
    \end{lem}

    \begin{proof}
        Since $G$ and the natural projection maps $\pi_N\colon G\to G/N$ form a cone over the $G/N$'s, then according to the universal property of projective limits there is a continuous group homomorphism $\alpha\colon G\to G'$. We only need to show that $\alpha$ is surjective.

        To prove surjectivity, first assume that $\alpha(G)$ is dense in $G'$. Then, since $G$ is compact, and the continuous image of a compact space is compact, we can conclude that $\alpha(G)$ is compact in $G'$. In a Hausdorff space compact subspaces are closed. Hence $\alpha(G)$ is closed and, therefore, is its own closure. The closure of a dense subset is the whole space, hence if $\alpha(G)$ is dense, it equals $G'$. So it suffices to prove that $\alpha(G)$ is dense in $G'$.

        To prove that $\alpha(G)$ is dense in $G'$ we will prove that it is not disjoint from any non-empty open set. If the $p_N\colon G'\to G/N$ are the projection maps from $G'$, a subbasis for the topoology on $G'$ is the set of subsets of the form $p^{-1}_N(S_N)$ for arbitrary subsets $S_N\subset G/N$. Finite intersections of such sets form a basis for the topology on $G'$. Since open sets in $G'$ are unions of such sets, it suffices for us to prove that no set in the basis is disjoint from $\alpha(G)$. Let
        $$p^{-1}_{N_1}(S_{N_1})\cap\cdots\cap p^{-1}_{N_j}(S_{N_r})$$
        be a member of the basis. Then
        $$M=N_1\cap\cdots\cap N_r$$
        is an open normal subgroup of $G$. Note that since every sequence $(x_N)\in G'$ must be compatible with the maps $\varphi_{MN_i}\colon G/N_i\to G/M$, the element $x_M$ fully determines the value of $x_{N_i}$ for every $1\leq i\leq r$. Therefore
        $$
        U = p^{-1}_M\bigg(\bigcap_{i=1}^{r} \varphi_{MN_i}(S_{N_i})\bigg)
        $$
        Now suppose $U$ is non-empty, \emph{i.e.} there is a $(x_N)\in G'$ such that
        $$x_M\in \bigcap_{i=1}^{r} \varphi_{MN_i}(S_{N_i})$$
        Since $\pi_M\colon G\to G/M$ is surjective there should be a $t\in G$ such that $\pi_M(t)=x_M$. Note that since the diagram
        \[\begin{tikzcd}
            G && {G'} \\
            \\
            && {G/M}
            \arrow["\alpha"', from=1-1, to=1-3]
            \arrow["{\pi_M}", from=1-1, to=3-3]
            \arrow["{p_M}"', from=1-3, to=3-3]
        \end{tikzcd}\]
        commutes according to the universal property of projective limits, we have $p_M(\alpha(t))=x_M$. Therefore, $\alpha(t)\in U$ and $U$ intersects $\alpha(G)$ as desired.
    \end{proof}

    \begin{lem}\label{lem:ons}
        Let $G$ be a compact, totally disconnected topological group. Then every neighborhood of identity contains an open normal subgroup.
    \end{lem}

    \begin{proof}
        Let $U$ be an open neighborhood of identity. The proof proceeds in three steps.

        \noindent\emph{Step 1. There is a compact, open neighborhood of identity $W\subset U$.}

        First, note that $G$ is Hausdorff since for any two points $x\neq y$ there should be open sets seperating them or otherwise $\{x,y\}$ would be connected and for topological groups satisfying $T_1$ is equivalent to being Hausdorff (\cite[Proposition 1-3]{FANF1999}). From this it follows that $G-U$ is compact, since in a Hausdorff space closed sets are compact.

        Now, let $Y$ be the intersection of all compact open neighborhoods of identity. The set $Y$ is connected according to \cite[Lemma 1-16]{FANF1999}. Therefore, $Y=\{e\}$. This implies that the complements of compact neighborhoods of identity cover $G-U$. Since $G-U$ is compact it follows from this that there are compact, open neighborhoods $K_1,\cdots,K_r$ of identity whose complements cover $G-U$. Let
        $$
        W=\bigcap_{i=1}^r K_i
        $$
        The set $W$ is an intersection of closed sets and is hence closed. Since $G$ is Hausdorff, $W$ must also be compact. Moreover, $W$ is a finite intersection of open sets and is therefore open. This completes Step 1.
        
        \noindent\emph{Step 2. There is a symmetric, open neighborhood of identity $V$ such that $WV\subset W$.}

        Recall that a set $V$ is called symmetric if $V^{-1}=V$. Now, consider the continuous map $\mu\colon W\times W \to G$ which is the restriction of the group operation on $G$ to $W \times W$. Since $W$ is open, $\mu^{-1}(W)$ must be open in $W \times W$. In the case of the product of finitely many topological spaces, the product topology conincides with the box topology and is generated by products of open sets. This means that
        $$ \mu^{-1} = \bigcup_{i \in I} U_i \times V_i $$
        for some index set $I$ such that all sets $U_i$ and $V_i$ are open in $W$. Since for every $w \in W$ we have $(w,e)\in\mu^{-1}(W)$ this implies that for every $w\in W$ there are open sets $U_w$ containing $w$ and $V_w$ containing $e$ such that $U_w\times V_w\subset \mu^{-1}(W)$. Moreover, since according to \cite[Proposition 1-1(ii)]{FANF1999} every neighborhood of identity contains a symmetric, open neighborhood of identity we may assume that all sets $V_w$ are symmetric. The sets $U_w$ clearly cover $W$ and since $W$ is compact a finite sub-cover $U_1,\cdots,U_r$ must exist. If $V_1,\cdots,V_r$ are the associated symmetric neighborhoods of identity, the set
        $$ V = \bigcap_{i=0}^r V_r $$
        is symmetric and open and the condition $WV\subset W$ is satisfied by construction.

        \noindent\emph{Step 3. There is an open subgroup $O$ contained in $W$.}

        It follows from the fact that $WV \subset W$ and $e \in W$ that $V \subset W$. By induction, it can be proved that $V^n \subset W$ for any integer $n \geq 1$. Now, let
        $$ O = \bigcup_{n\geq 1} V^n $$
        It is clear that $O$ is open and is closed under inverses and multiplication.

        \noindent\emph{Step 4. There is an open normal subgroup $N$ of $G$ contained in $W$.}
        
        Since $O$ is open in $G$, $G/O$ (considered simply as a quotient topological space) has the discrete topology. Since $G/O$ is moreover compact, it follows that it is finite. Hence, $O$ has finitely many cosets in $G$. Let these cosets be represented by $x_1O,\cdots,x_sO$. By elementary group theory, it follows that the conjugates of $O$ are all of the form $x_iOx_i^{-1}$ for some $1\leq i\leq s$. Let
        $$ N = \bigcap_{i=1}^s x_iOx_i^{-1} $$
        It is clear that $N$ is open. Moreover, after doing some group theory it becomes evident that $N$ is a normal subgroup of $G$. Finally, since $O$ is a conjugate of itself and, therefore, $N\subset O$ we also have $N\subset W$ as desired.
    \end{proof}

    \begin{lem}\label{lem:chb}
        Let $f\colon X\to Y$ be a continuous bijective mapping of topological spaces where $X$ is compact and $Y$ is Hausdorff. Then $f$ is a homeomorphism.
    \end{lem}

    \begin{proof}
        We need to prove that $f^{-1}$ is continuous, \emph{i.e.} for every open set $U\subset X$, the set $(f^{-1})^{-1}(U)=f(U)$ is open in $Y$. For every open set $U\subset X$, its complement $U^c$ is closed in $X$ and hence compact. The continuous image of a compact set is compact and therefore $f(U^c)$ is compact in $Y$. In a Hausdorff space, compact sets are closed and therefore $f(U^c)$ is closed in $Y$. This implies that $f(U)=f(U^c)^c$ is open in $Y$.
    \end{proof}

    Now we are in a position to prove the following:

    \begin{thm}
        A topological group is profinite if and only if it is compact and totally disconnected.        
    \end{thm}

    \begin{proof}
        One side follows from Propositions \ref{prop:top}\ref{prop:top:compact} and \ref{prop:tod}. For the other side let $G$ be a compact, totally disconnected topological group. From Lemma \ref{lem:sur} we know that there is a continuous surjective group homomorphism $\alpha\colon G\to\varprojlim G/N$. By Lemma \ref{lem:chb}, it suffices to prove that $\alpha$ is injective, \emph{i.e.} it has trivial kernel. Note that
        $$
        \textup{ker }\alpha \subset \bigcup_{N\in\nor} N \subset \bigcup_{\substack{U \text{ open} \\ \text{nbhd. of } e}} U = \{e\}
        $$
        where the second inclusion follows from Lemma \ref{lem:ons} and the equality is an implication of $G$ being totally disconnected. (Suppose the intersection of all open neighborhoods of identity includes an element $x\neq e$. Then $\{e,x\}$ would be a connected subspace of $G$ that contains $e$.) Therefore, $\textup{ker }\alpha=\{e\}$ and $\alpha$ is injective as desired.
    \end{proof}

    \section{Subgroups of Profinite Groups}

    \begin{thm}
        Let $G$ be a profinite group and $H$ an open subgroup of $G$. Then $G/H$ is finite.  
    \end{thm}

    \begin{proof}
        Let $H$ be open. Then $G/H$ with the discrete topology allows the canonical projection map to be continuous. Therefore, $G/H$ has the discrete topology and being compact, is finite.
    \end{proof}

    \begin{rem}
        It is claimed in \cite*[Proposition 1-18]{FANF1999} that in a profinite group subgroups of finite index are necessarily open. This claim is wrong. For a counterexample see \cite*[Proposition 7.29]{milneFT}.
    \end{rem}

    \begin{thm}
        Let $G$ be a profinite group and $H$ a subgroup of $G$. The following are equivalent:
        \begin{enumerate}[label=\emph{(\roman*)}]
            \item $H$ is closed.
            \item $H$ is profinite.
            \item $H$ is the intersection of a family of open subgroups.
        \end{enumerate}
    \end{thm}

    \begin{proof}
        (i) to (ii). Suppose $H$ is closed. Then $H$ is compact, since every closed set in a compact space is compact. Moreover $H^\circ$ equals $\gcirc$ and hence $H$ is totally disconnected. Theorefore, $H$ is profinite.

        (ii) to (i). If $H$ is compact, then it is closed in $G$ since $G$ is Hausdorff.

        (iii) to (i). Every open subgroup of a topological group is also closed, since all its cosets are open and its complement is a union of a family of its cosets. Therefore, if $H$ is the intersection of a family of open sets it is \emph{ipso facto} the intersection of a family of closed sets and hence closed.

        (i) to (iii). Let $\nor$ be the set of all normal subgroups of $G$. For any $N\in H$, the set $NH$ is a subgroup of $G$. (Note that $nhn'h'=n(hn'h^{-1})hh'$ and $(nh)^{-1}=(h^{-1}nh)h^{-1}$.) The subgroup $NH$ is moreover open since it is the union of sets of the form $Nh$ for $h\in H$ which are open. We claim
        $$ H = \bigcap_{N\in\nor} NH $$
        One inclusion is trivial. For the other side let $x$ be a member of the intersection. We will prove that every open neighborhood of $x$ intersects $H$ and therefore $x$ lies in the closure of $H$ which is $H$ itself. Now let $U$ be an open neighborhood of $x$. Then $Ux^{-1}$ is an open neighborhood of identity and hence, by Lemma \ref{lem:ons}, contains an open normal subgroup of $G$ which we will call $N_x$. Now $x$ is both a member of $N_xx$ (by construction) and $N_xH$ (by the assumption the $x$ in a member of the intersection). Therefore, $N_xx=N_xh$ for some $h\in H$ which implies $h\in N_xx\subset U$ as desired.
    \end{proof}

    \begin{thm}
        Let $G$ be a profinite group and $H$ a closed subgroup of $G$. Then $G/H$ is compact and totally disconnected, and hence profinite, provided it is a group.
    \end{thm}

    \begin{proof}
        It is evident that $G/H$ is compact. To prove that $G/H$ is totally disconnected we first need to prove that it is Hausdorff, but this follows from \cite[Proposition 1-3 and Proposition 1-4(iii)]{FANF1999}. Now to prove that $G/H$ is totally disconnected it suffices to prove that for any $x\in G/H$ such that $x\neq e$ there is a closed and open set in $G/H$ that separates $x$ from $e$. Since $G/H$ is Hausdorff there is an open set $U$ in $G/H$ that contains $e$ but does not contain $x$. Let $\rho\colon G\to G/H$ be the canonical projection and consider the open set $\rho^{-1}(U)\subset G$. According to Lemma \ref{lem:ons} there is an open subgroup $V$ of $G$ that is contained in $U$. Moreover, the subgroup $V$ is closed since its complement is the union of its nontrivial cosets which are open. Since $G$ is Hausdorff this implies that $V$ is compact. Now, $\rho(V)$ is open since $\rho$ is an open map (see \cite[Proposition 1-4(ii)]{FANF1999}) and is closed since the continuous image of a compact set is compact and compact and closed are equivalent in a Hausdorff space. Note that $e\in\rho(V)$ but $x\not\in\rho(V)\subset U$ which proves our assertion.
    \end{proof}

    \printbibliography

\end{document} 