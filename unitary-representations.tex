\documentclass[12pt]{article}

\usepackage{enumitem}
\usepackage[hidelinks]{hyperref}

\usepackage[backend=bibtex, style=numeric, citestyle=numeric-comp]{biblatex}

\usepackage{amsmath,amssymb,amsfonts,amsthm}

\usepackage{tikz-cd}

\newtheorem{thm}{Theorem}[section]
\newtheorem{prop}[thm]{Proposition}
\newtheorem{lem}[thm]{Lemma}
\newtheorem{corollary}[thm]{Corollary}

\theoremstyle{definition}
\newtheorem{dfn}[thm]{Definition}
\newtheorem{exm}[thm]{Example}

\theoremstyle{remark}
\newtheorem{rem}[thm]{Remark}
\newtheorem{notation}[thm]{Notation}

\newcommand{\catvar}[1]{\mathcal{#1}}
\newcommand{\CC}{\catvar{C}}
\newcommand{\DD}{\catvar{D}}
\newcommand{\PP}{\catvar{P}}
\newcommand{\II}{\catvar{I}}
\newcommand{\JJ}{\catvar{J}}

\newcommand{\Comp}{\mathbb{C}}
\newcommand{\Real}{\mathbb{R}}

\newcommand{\catname}[1]{\textup{\sffamily\textbf{#1}}}
\newcommand*{\Set}{\catname{Set}}
\newcommand*{\Grp}{\catname{Grp}}
\newcommand*{\Top}{\catname{Top}}
\newcommand*{\TopGrp}{\catname{TopGrp}}

\newcommand{\zee}{\mathbb{Z}}
\newcommand{\zeehat}{\hat{\mathbb{Z}}}

\newcommand*{\nor}{\mathcal{N}}

\newcommand{\Obj}[1]{\textup{Obj}(#1)}
\newcommand{\Mor}[1]{\textup{Mor}(#1)}
\newcommand{\Hom}[2]{\textup{Hom}(#1,#2)}
\newcommand{\End}[1]{\textup{End}(#1)}

\newcommand{\gcirc}{G^\circ}

\addbibresource{profinite-groups.bib}

\title{Unitary Representations}
\author{Navid Rashidian}
\date{}

\begin{document}

    \maketitle

    \section{Recap}

    \begin{dfn}
        A normed real (complex) vector space which is complete with respect to the norm metric is called a \emph{real (complex) Banach space}.
    \end{dfn}

    \begin{dfn}
        Let $A,B$ be Banach spaces defined over the same field. A linear transformation $T\colon A\to B$ is a \emph{bounded operator} if there is constant $c\in\Real$ such that for every $v\in A$ we have $\|T(v)\|\leq c\|v\|$.
    \end{dfn}

    \begin{prop}
        A linear transformation between two Banach spaces is bounded if and only if it is continuous.
    \end{prop}

    \begin{notation}
        The set of all bounded operators $T\colon A\to B$ between two Banach spaces is denoted by $\Hom{A}{B}$. In the particular case in which $A=B$, we have $\End{A}=\Hom{A}{A}$.
    \end{notation}

    \begin{dfn}
        Let $H$ be a complex vector space. A function $\langle\ |\ \rangle\colon H\times H\to\Comp$ is \emph{positive definite Hermitian form} if it satifies the following conditions:
        \begin{enumerate}
            \item Positive definiteness: $\langle u|r\rangle$ is a non-negative integer for every $u\in H$ and is equal to zero if and only if $u$ equals zero.
            \item Conjugate symmetry: For every $u,v\in H$, $\langle u|v\rangle = \overline{\langle v|u \rangle}$.
            \item Linearity in the first variable: For every $\lambda,\mu\in\Comp$ and $u,v,w\in H$, $\langle\lambda u+\mu v|w\rangle=\lambda\langle u|w\rangle+\mu\langle u|v\rangle$.
        \end{enumerate}
    \end{dfn}
    
    \begin{prop}
        Let $H$ be a complex vector space and $\langle\ |\ \rangle$ a positive definite Hermitian form. Then the function $\|v\|=\sqrt{\langle v|v \rangle}$ defines a norm on $H$.
    \end{prop}

    \begin{dfn}
        A complex vector space $H$ together with a positive definite Hermitian form is called a \emph{pre-Hilbert space}. If $H$ is complete with respect to the metric associated with the norm defined by the Hermitian form, it is called a \emph{Hilbert space}.
    \end{dfn}

    \begin{rem}
        A Hilbert space is a complex Banach space and therefore a topological vector space.
    \end{rem}

    \begin{prop}
        Let $H$ be a Hilbert space. For every $T\in\End{A}$ there is a $T^*\in\End{H}$ called the \emph{adjoint of $T$} such that
        $$
        \langle Tv|u \rangle = \langle v|T^*u \rangle
        $$
        for every $u,v\in H$.
    \end{prop}

    \section{Unitary Representations}

    % TODO
    \begin{prop}
        The Hermitian form on a pre-Hilbert space is a continuous linear functional.
    \end{prop}

    \begin{prop}
        Let $H$ be a pre-Hilbert space and $\hat H$ its completion. Also, let $\lambda\colon H\to\mathbb{C}$ be a continuous linear functional. Then $\lambda$ extends uniquely to $\hat H$.
    \end{prop}

    \begin{proof}
        Let $\hat \lambda((x_i)_{i\in\mathbb{N}})=\lim_{n\to\infty}\lambda(x_i)$. Since $\lambda$ is continuous, this is well-defined. Its uniqueness also follows from the fact that we suppose $\hat \lambda$ to be continuous. It is routine to check that $\hat \lambda$ respects addition and multiplication by scalars.
    \end{proof}

    \begin{corollary}
        Let $H$ be a pre-Hilbert space and $\hat{H}$ its completion. Then the Hermitian form on $H$ can be extended to $\hat{H}$ and $\hat{H}$ will be complete with respect to the metric associated to the Hermitian form.
    \end{corollary}

    \begin{prop}
        Let $T\in\End{H}$. There is a unique $\hat T\in\End{\hat H}$ such that $\hat T|_H = T$.
    \end{prop}

    \begin{corollary}
        Let $H$ be a pre-Hilbert space and $T\in\End{H}$. There is a $T^*\in\End{H}$ such that
        $$
        \langle Tv|u \rangle = \langle v|T^*u \rangle
        $$
        for every $u,v\in H$.
    \end{corollary}

    \begin{dfn}
        Let $H$ be a pre-Hilbert space. A bounded linear operator $T\in\End{H}$ is called \emph{pre-unitary} if we have
        $\langle x|y \rangle = \langle T(x)|T(y) \rangle$
        for every $x,y\in H$.
    \end{dfn}

    \begin{prop}
        A bounded linear operator $T\in\End{H}$ is pre-unitary if and only if $T^*T=\text{id}_H$.
    \end{prop}

    \begin{proof}
        Assume $T^*T=1_H$. Then $\langle Tx|Ty \rangle = \langle x |T^*Ty \rangle = \langle x|y \rangle$. For the other side, let $A=T^*T-\text{id}_H$. Obviously, for every $x\in H$ we have
        $$
        \langle x | Ax \rangle = \langle x | T^*Tx - x \rangle = \langle x | T^*Tx \rangle - \langle x | x \rangle = 0
        $$
        We now proceed to prove that any linear transformation $A$ with the property that for every $x\in H$ we have $\langle x | Ax \rangle$ is identically zero. For this let $x,y\in H$ and note that
        \begin{align*}
            \langle x+y | A(x+y) \rangle &= \langle x | Ax \rangle + \langle x | Ay \rangle + \langle y | Ax \rangle + \langle y | Ay \rangle \\
            &= \langle x | Ay \rangle + \langle y | Ax \rangle \\
            &= 0
        \end{align*}
        and
        \begin{align*}
            \langle x + iy | A(x + iy) \rangle &= \langle x + iy | Ax + iAy \rangle \\
            &= \langle x | Ax \rangle + i \langle x | Ay \rangle - i \langle y | Ax \rangle + \langle y | Ay \rangle \\
            &= i (\langle x | Ay \rangle - \langle y | Ax \rangle) \\
            &= 0
        \end{align*}
        which imply $\langle x|Ay \rangle = 0$ for every $x,y\in H$. In particular $\langle Ax|Ax \rangle = 0$ which implies $Ax = 0$.
        
    \end{proof}

    \printbibliography

\end{document} 