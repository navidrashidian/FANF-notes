\documentclass[12pt]{article}

\usepackage{enumitem}
\usepackage[hidelinks]{hyperref}

\usepackage[backend=bibtex, style=numeric, citestyle=numeric-comp]{biblatex}

\usepackage{amsmath,amssymb,amsfonts,amsthm}
\usepackage{mathrsfs}

\usepackage{tikz-cd}

\newtheorem{theorem}{Theorem}[section]
\newtheorem{prop}[theorem]{Proposition}
\newtheorem{lem}[theorem]{Lemma}
\newtheorem{corollary}[theorem]{Corollary}

\theoremstyle{definition}
\newtheorem{dfn}[theorem]{Definition}
\newtheorem{example}[theorem]{Example}

\theoremstyle{remark}
\newtheorem{rem}[theorem]{Remark}
\newtheorem{notation}[theorem]{Notation}

\newcommand{\catvar}[1]{\mathcal{#1}}
\newcommand{\CC}{\catvar{C}}
\newcommand{\DD}{\catvar{D}}
\newcommand{\PP}{\catvar{P}}
\newcommand{\II}{\catvar{I}}
\newcommand{\JJ}{\catvar{J}}

\newcommand{\Comp}{\mathbb{C}}
\newcommand{\Real}{\mathbb{R}}

\newcommand{\catname}[1]{\textup{\sffamily\textbf{#1}}}
\newcommand*{\Set}{\catname{Set}}
\newcommand*{\Grp}{\catname{Grp}}
\newcommand*{\Top}{\catname{Top}}
\newcommand*{\TopGrp}{\catname{TopGrp}}

\newcommand{\zee}{\mathbb{Z}}
\newcommand{\zeehat}{\hat{\mathbb{Z}}}

\newcommand*{\nor}{\mathcal{N}}

\newcommand{\Obj}[1]{\textup{Obj}(#1)}
\newcommand{\Mor}[1]{\textup{Mor}(#1)}
\newcommand{\Hom}[2]{\textup{Hom}(#1,#2)}
\newcommand{\End}[1]{\textup{End}(#1)}

\newcommand{\gcirc}{G^\circ}

\addbibresource{bibliography.bib}

\title{Unitary Representations}
\author{Navid Rashidian}
\date{}

\begin{document}

    \maketitle

    \section{Recap}

    \subsection{Hilbert Spaces}

    \begin{dfn}
        A normed real (complex) vector space which is complete with respect to the norm metric is called a \emph{real (complex) Banach space}.
    \end{dfn}

    \begin{dfn}
        Let $A,B$ be Banach spaces defined over the same field. A linear transformation $T\colon A\to B$ is a \emph{bounded operator} if there is constant $c\in\Real$ such that for every $v\in A$ we have $\|T(v)\|\leq c\|v\|$.
    \end{dfn}

    \begin{dfn}
        Let $A$ be a Banach space endowed with a multiplication such that $A$ with its addition and multiplication form a ring. If for every scalar $\lambda$ we have $\lambda(ab)=(\lambda a)b$, and moreoever we have $\|ab\|\leq \|a\|\|b\|$, then $A$ is called a \emph{Banach algebra}. If, moreoever, $A$ has a multiplicative identity (denoted by $1_A$) it is called \emph{unital}.
    \end{dfn}

    \begin{dfn}
        Let \( A, B \) be Banach spaces defined over the same field. Then a linear transformation \( T\colon A\to B \) is called bounded if there is a real constant \( c \) such that for every \( a \in A \), we have
        \begin{equation} \label{eq:bounded_inequality}
            \|T(a)\| \leq c \|a\|.
        \end{equation}
    \end{dfn}

    \begin{prop}
        A linear transformation between two Banach spaces is bounded if and only if it is continuous.
    \end{prop}

    \begin{proof}
        See \cite[Ch. II, Prop. 1.1]{Conway1985}.
    \end{proof}

    \begin{notation}
        The set of all bounded operators $T\colon A\to B$ between two Banach spaces is denoted by $\Hom{A}{B}$. In the particular case in which $A=B$, we have $\End{A}=\Hom{A}{A}$.
    \end{notation}

    \begin{dfn}
        Let $H$ be a complex vector space. A function $\langle\ |\ \rangle\colon H\times H\to\Comp$ is \emph{positive definite Hermitian form} if it satifies the following conditions:
        \begin{enumerate}
            \item Positive definiteness: $\langle u|r\rangle$ is a non-negative integer for every $u\in H$ and is equal to zero if and only if $u$ equals zero.
            \item Conjugate symmetry: For every $u,v\in H$, $\langle u|v\rangle = \overline{\langle v|u \rangle}$.
            \item Linearity in the first variable: For every $\lambda,\mu\in\Comp$ and $u,v,w\in H$, $\langle\lambda u+\mu v|w\rangle=\lambda\langle u|w\rangle+\mu\langle u|v\rangle$.
        \end{enumerate}
    \end{dfn}
    
    \begin{prop}
        Let $H$ be a complex vector space and $\langle\ |\ \rangle$ a positive definite Hermitian form. Then the function $\|v\|=\sqrt{\langle v|v \rangle}$ defines a norm on $H$.
    \end{prop}

    \begin{dfn}
        A complex vector space $H$ together with a positive definite Hermitian form is called a \emph{pre-Hilbert space}. If $H$ is complete with respect to the metric associated with the norm defined by the Hermitian form, it is called a \emph{Hilbert space}.
    \end{dfn}

    \begin{rem}
        A Hilbert space is a complex Banach space and therefore a topological vector space.
    \end{rem}

    \begin{prop}
        The set \( \End{H} \) for a Hilbert space \( H \) is a complex Banach algebra with pointwise addition and composition, and the norm of \( T \) defined as the smallest constant \( c \) that makes inequality \eqref{eq:bounded_inequality} true.
    \end{prop}

    \begin{prop}
        Let $H$ be a Hilbert space. For every $T\in\End{A}$ there is a $T^*\in\End{H}$ called the \emph{adjoint of $T$} such that
        $$
        \langle Tv|u \rangle = \langle v|T^*u \rangle
        $$
        for every $u,v\in H$.
    \end{prop}

    \begin{prop}
        The adjoint operator has the following properties for every $T\in\End{T}$:
        \begin{enumerate}[label=(\roman*)]
            \item Period Two: $T^{**}=T$.
            \item Conjugate Linearity: $(\lambda_1 T_1+\lambda_2 T_2)^*=\bar{\lambda_1}T_1^*+\bar{\lambda_2}T_2^*$.
            \item Antimultiplicativity: $(T_1T_2)^*=T_2^*T_1^*$.
            \item $\|TT^*\|=\|T\|^2$.
        \end{enumerate}
    \end{prop}

    \begin{dfn}
        Let $A$ be a complex algebra. An operator $a\mapsto a^*$ is called an \emph{involution} if it has period two, is conjugate linear and is antimultiplicative. If, moreover, $A$ is a Banach algebra and we have $\|aa^*\|=\|a\|^2$ for every $a\in A$ we call $A$ a $C^*$-algebra. A function between complex algebras $A$ and $B$ that preserves the complex algebra and the $*$-operator is called a $*$-isomorphism.
    \end{dfn}
    
    \begin{example}
        The complex numbers form a $C^*$-algebra with the involution operator defined as complex conjugation.
    \end{example}

    \begin{dfn}
        Let $T\in\End{H}$ for a Hilbert space $H$. The transformation $T$ is called \emph{normal} if it commutes with $T^*$. It is called \emph{self-adjoint} or \emph{Hermitian} if $T^*=T$. It is called \emph{unitary} if $T^*=T^{-1}$. A subalgebra of $\End{H}$ is called \emph{self-adjoint} if it is closed under $^*$.
    \end{dfn}

    \begin{dfn}
        Let $A$ be a complex Banach algebra and $a\in A$. Then, \emph{the spectrum of a} denoted by $\textup{sp}(a)$ is defined as
        $$
        \textup{sp}(a) = \{\lambda\in\Comp:\lambda 1_A-a\not\in A^\times\}
        $$
    \end{dfn}

    \begin{prop}
        Let $A$ be a complex Banach algebra and $a\in A$. The spectrum of $a$ is a $C^*$-algebra with involution defined as complex conjugation.
    \end{prop}

    \begin{prop}
        Assume $T\in\End{H}$ is normal. Let $A_T$ be the smallest closed, self-adjoint, unital subalgebra of $\End{H}$ that contains $T$. Then $A_T$ is the closure of the subalgebra generated by $T$ and $T^*$ and is commutative.
    \end{prop}

    \begin{theorem}[First Spectral Theorem]
        Assume $T\in\End{H}$ is normal. Then there exists an isometric $*$-isomorphism $\Phi\colon \mathscr{C}(\textup{sp}(T))\to A_T$ such that $\Phi(\iota_{\textup{sp}(a)})=T$.
    \end{theorem}
    

    \subsection{Group Representations}

    \begin{dfn}
        Let $G$ be a group and $V$ a complex vector space. Let $\textup{Aut}(V)$ denote the group of isomorphisms of $V$ onto itself. A homomorphism $\rho\colon G\to\textup{GL}(V)$ is called \emph{an abstract representation of $G$ in $V$}.
    \end{dfn}

    \begin{dfn}
        Let $G$ be a group, $V,V'$ vector spaces and $\rho,\rho'$ representations of $G$ in, respectively, $V$ and $V'$. The representations $\rho$ and $\rho'$ are called \emph{equivalent} if there is an isomorphism $T\colon V \to V'$ such that
        $ T\rho_g=\rho'_gT $
        for every $g \in G$.
    \end{dfn}

    \begin{dfn}
        Let $\rho$ and $\rho'$ be representations of a group $G$ on, respectively, $V$ and $W$. A \emph{$G$-linear map between $\rho$ and $\rho'$} is a linear transformation $\phi\colon V\to W$ such that for every $g\in G$ we have $\phi(\rho_g(v))=\rho'_g(\phi(v))$ for all $v\in V$.
    \end{dfn}

    \begin{dfn}
        Let $\rho\colon G \to \textup{Aut}(V)$ be a representation of a group $G$ in a vector space $V$. A subspace $W$ of $V$ is called \emph{stable under the action of $G$} if for every $g$, $x \in W$ implies $\rho_g(x) \in W$.
    \end{dfn}

    \begin{dfn}
        A representation $\rho\colon G \to \textup{Aut}(V)$ of a group $G$ in a vector space $V$ is called \emph{irreducible} if $V$ is not $0$ and if no subspace of $V$, save $0$ and $V$, is stable under $G$.
    \end{dfn}

    \begin{dfn}
        A field $k$ subject to a certain topology is called a \emph{topological field} if both addition and multiplication are continuous with respect to the product topology on $k\times k$. A vector space $V$ subject to a certain topology over a topological field is called a \emph{topological vector space} if its underlying group is a topological group and scalar multiplication is continuous with respect to the product topology on $k \times V$
    \end{dfn}

    \begin{dfn}
        Let $G$ be a locally compact topological group and $V$ a locally convex topological vector space over the complex numbers. An abstract representation $\rho\colon G\to\textup{Aut}(V)$ is called a \emph{topological representation} if the map $(g,x)\mapsto\rho_g(x)$ is continuous with respect to the product topology on $G\times V$.
    \end{dfn}

    \begin{rem}
        The topology induced by a norm is generated by open balls and open balls are convex (see \cite[Ch. IV, \S 1]{Conway1985}). Therefore, for any normed space $V$ we could talk of representations of groups in $V$.
    \end{rem}

    \section{Unitary Representations}

    % TODO
    \begin{prop}
        The Hermitian form on a pre-Hilbert space is a continuous linear functional.
    \end{prop}

    \begin{prop}
        Let $H$ be a pre-Hilbert space and $\hat H$ its completion. Also, let $\lambda\colon H\to\mathbb{C}$ be a continuous linear functional. Then $\lambda$ extends uniquely to $\hat H$.
    \end{prop}

    \begin{proof}
        Let $\hat \lambda((x_i)_{i\in\mathbb{N}})=\lim_{n\to\infty}\lambda(x_i)$. Since $\lambda$ is continuous, this is well-defined. Its uniqueness also follows from the fact that we suppose $\hat \lambda$ to be continuous. It is routine to check that $\hat \lambda$ respects addition and multiplication by scalars.
    \end{proof}

    \begin{corollary}
        Let $H$ be a pre-Hilbert space and $\hat{H}$ its completion. Then the Hermitian form on $H$ can be extended to $\hat{H}$ and $\hat{H}$ will be complete with respect to the metric associated to the Hermitian form.
    \end{corollary}

    \begin{prop}
        Let $T\in\End{H}$. There is a unique $\hat T\in\End{\hat H}$ such that $\hat T|_H = T$.
    \end{prop}

    \begin{corollary}
        Let $H$ be a pre-Hilbert space and $T\in\End{H}$. There is a $T^*\in\End{H}$ such that
        $$
        \langle Tv|u \rangle = \langle v|T^*u \rangle
        $$
        for every $u,v\in H$.
    \end{corollary}

    \begin{dfn}
        Let $H$ be a pre-Hilbert space. A bounded linear operator $T\in\End{H}$ is called \emph{pre-unitary} if we have
        $\langle x|y \rangle = \langle T(x)|T(y) \rangle$
        for every $x,y\in H$. For Hilbert spaces $H$ and $H'$ an isomorphism $T\colon H\to H'$ is called \emph{pre-unitary} if the same condition holds for every $x,y\in H$. 
    \end{dfn}

    \begin{prop}
        A bounded linear operator $T\in\End{H}$ is pre-unitary if and only if $T^*T=\textup{id}_H$.
    \end{prop}

    \begin{proof}
        Assume $T^*T=1_H$. Then $\langle Tx|Ty \rangle = \langle x |T^*Ty \rangle = \langle x|y \rangle$. For the other side, let $A=T^*T-\textup{id}_H$. Obviously, for every $x\in H$ we have
        $$
        \langle x | Ax \rangle = \langle x | T^*Tx - x \rangle = \langle x | T^*Tx \rangle - \langle x | x \rangle = 0
        $$
        We now proceed to prove that any linear transformation $A$ with the property that for every $x\in H$ we have $\langle x | Ax \rangle$ is identically zero. For this let $x,y\in H$ and note that
        \begin{align*}
            \langle x+y | A(x+y) \rangle &= \langle x | Ax \rangle + \langle x | Ay \rangle + \langle y | Ax \rangle + \langle y | Ay \rangle \\
            &= \langle x | Ay \rangle + \langle y | Ax \rangle \\
            &= 0
        \end{align*}
        and
        \begin{align*}
            \langle x + iy | A(x + iy) \rangle &= \langle x + iy | Ax + iAy \rangle \\
            &= \langle x | Ax \rangle + i \langle x | Ay \rangle - i \langle y | Ax \rangle + \langle y | Ay \rangle \\
            &= i (\langle x | Ay \rangle - \langle y | Ax \rangle) \\
            &= 0
        \end{align*}
        which imply $\langle x|Ay \rangle = 0$ for every $x,y\in H$. In particular $\langle Ax|Ax \rangle = 0$ which implies $Ax = 0$.
        
    \end{proof}

    \begin{dfn}
        Let $\rho$ be a representation of a locally compact group $G$ on a pre-Hilbert space $H$. We call $\rho$ \emph{pre-unitary} if for every $g\in G$ topological automorphism $\rho_g$ is pre-unitary. If this happens the Hermitian form on $H$ is called \emph{invariant under $\rho(G)$}.
    \end{dfn}

    \begin{dfn}
        Let $G$ be a group and $H,H'$ pre-Hilbert spaces. we call topological representations $\rho,\rho'$ of $G$ in, respectively, $H$ and $H'$ \emph{pre-unitarily equivalent} if there is a pre-unitary topological isomorphism $T\colon H\to H'$ such that $T\rho_g=\rho'_gT$ for every $g\in G$.
    \end{dfn}

    \printbibliography

\end{document} 